
	
\usetikzlibrary{calc,patterns.meta}
% To provide total amount of sections throughout the document
\usepackage{totcount}
% Registers de total amount of sections (see https://tex.stackexchange.com/a/192506/141947)
\regtotcounter{section}
% To be able to refer to sections when needed
\usepackage{nameref}
% Redefinition of the \section command so that each one is labeled \label{sec:n} where n is its index 
\let\oldsection\section
\renewcommand{\section}[2][\relax]{%
	\ifx#1\relax
	\oldsection{#2}%
	\else
	\oldsection[#1]{#2}%
	\fi%
	\label{sec:\thesection}%
}

% Definition of custom colors based on the initial figure of the bar by the OP
\definecolor{color1}{HTML}{002868}%original = blue 57AED1
\definecolor{color2}{HTML}{BF0A30}%original = green 8BC53F
\definecolor{mygray}{HTML}{EEEEEE}

% Definition of custom tikz styles in order to ease readability
\tikzset{
	% Bar style (Argument : color)
	sectionbar/.style={
		% Filling with one color as a preaction, in order to avoid reset by the pattern color
		preaction={fill=#1!70},
		% Application of the line pattern on to of the fill
		pattern={Lines[angle=45,distance={6pt},line width=3pt]},pattern color=#1
	},
	% Node style (Arguments : color, section number)
	sectionnode/.style 2 args={
		fill=#1,
		draw=white,
		thick,
		circle,
		text=white,
		radius=10pt,
		% Display of the section name below the cicle
		label={[text=#1]below:\nameref{sec:#2}},
	}
}


% Actual definition of the colorbar based on Gonzalo Medina's initial proposal
\makeatletter
\def\pbar@progressbar{} % the progress bar
\newcount\pbar@tmpcnta% auxiliary counter
\newcount\pbar@tmpcntb% auxiliary counter
\newdimen\pbar@pbht %progressbar height
\newdimen\pbar@pbwd %progressbar width
\newdimen\pbar@tmpdim % auxiliary dimension
\pbar@pbwd=\linewidth
\pbar@pbht=4pt

% The progress bar
\def\pbar@progressbar{%
	\pbar@tmpcnta=\value{section} % tmpcnta stores the section number
	\pbar@tmpcntb=\totvalue{section} % tmbcountb sotres the total amount of sections
	\advance\pbar@tmpcntb by 1 % tmbcountb is advanced by 1 in order to have the last bar segment after the last node
	
	\begin{tikzpicture}[very thin]
		% Clipping scope to avoid tests for the bar dimensions
		\begin{scope}
			% Clipping path
			\path[rounded corners=2pt,clip] (0pt,{-\pbar@pbht/2}) rectangle (\pbar@pbwd,{\pbar@pbht/2});
			% Gray bar (from 0 to last section)
			\path[sectionbar=mygray] (0pt,{-\pbar@pbht/2}) rectangle (\linewidth,{\pbar@pbht/2});
			% Blue bar (from 0 to the current section)
			\path[sectionbar=color1] (0pt,{-\pbar@pbht/2}) rectangle ({(\pbar@tmpcnta-0.5)*\linewidth/\pbar@tmpcntb},{\pbar@pbht/2});
			% Green bar (from current to next section)
			\path[sectionbar=color2] ({(\pbar@tmpcnta-0.5)*\linewidth/\pbar@tmpcntb},{-\pbar@pbht/2}) rectangle ({(\pbar@tmpcnta+0.5)*\linewidth/\pbar@tmpcntb},{\pbar@pbht/2});
		\end{scope}
		% Drawing of the nodes on top of the bars, based on the number of the current section
		\foreach \secnumber in {1,...,\totvalue{section}}{
			% Number is lower, section is past, blue color
			\ifnum\secnumber<\pbar@tmpcnta
			\node[sectionnode={color1}{\secnumber}] at ({(\secnumber-0.5)*\linewidth/\pbar@tmpcntb},0) {\strut\secnumber};
			\fi
			% Number is equal, section is current, green color
			\ifnum\secnumber=\pbar@tmpcnta
			\node[sectionnode={color2}{\secnumber}] at ({(\secnumber-0.5)*\linewidth/\pbar@tmpcntb},0) {\strut\secnumber};
			\fi
			% Number is larger, to be done section, gray color
			\ifnum\secnumber>\pbar@tmpcnta
			\node[sectionnode={mygray}{\secnumber}] at ({(\secnumber-0.5)*\linewidth/\pbar@tmpcntb},0) {\strut\secnumber};
			\fi
		}
	\end{tikzpicture}%
}

\addtobeamertemplate{headline}{}
{%
	\begin{beamercolorbox}[wd=\paperwidth,ht=10ex,center,dp=1ex]{white}%
		\pbar@progressbar%
	\end{beamercolorbox}%
}
\makeatother