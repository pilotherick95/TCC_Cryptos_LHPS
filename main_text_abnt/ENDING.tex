% !TeX encoding = UTF-8
% !TeX spellcheck = pt_BR

\chapter{Considera\c{c}\~oes Finais}

\begin{flushright}
	\begin{minipage}{8cm}
		\textit{	\noindent 
			\begin{flushright}
				``Invista seu dinheiro no comércio exterior, e um dia desses você terá lucro. %Colocai teus investimentos em vários lugares - muitos lugares, de fato - pois nunca saberás que tipo de azar terás neste mundo.'' 
				'' - Ecclesiastes 11:1, Good News Bible (GNB) (tradução livre)
		\end{flushright}}
		\vspace{1cm} 
	\end{minipage}
	\footnote{Citação original do inglês: ``Invest your money in foreign trade, and one of these days you will make a profit.'' %Put your investments in several places — many places, in fact — because you never know what kind of bad luck you are going to have in this world.''
		 \cite{BIBLE2}}.
\end{flushright}


Neste último capítulo buscou-se trazer os resultados obtidos tomando nota os objetivos gerais e específicos. Tem-se que os resultados aqui obtidos podem ser classificados como satisfatórios, 
de tal forma que todos os objetivos estabelecidos foram devidamente cumpridos. Abaixo, algumas rápidas
considerações.\\

\section{Conclusões}
A tecnologia traz seus avanços a cada dia que sucede e estes vêm se tornando ferramentas cada vez mais indispensáveis. Um exemplo seria o acesso ao sistema interglobal de redes computadorizadas\footnote{Popularmente conhecida como internet}. Por causa disto, há entre vós muitos artefatos que se moldaram devido ao seu surgimento e muitos outros que só puderam existir devido a sua concepção. \\
O presente instrumento procurou trazer de forma simplificada que as criptomoedas são um exemplo destes artefatos que só puderam existir e permanecer aos dias atuais devido a conectividade global, além de toda a história que antecedeu todo este processo. \\
A evolução das moedas nos mostrou que estas perduraram por um grande número de formas ao decorrer da história, desde o escambo aos preciosos metais pesados até a moeda impressa, e finalmente aos cartões de crédito e débito até aos criptoativos, sempre se modelando para satisfazer às necessidades da expansão da sociedade, e claro, sempre buscando a aceitação do coletivo, desempenhando suas múltiplas funções\footnote{Suas funções são estas:\begin{enumerate}[label=\Roman*]
		\item Servir como meio de troca
		\item Atuar como unidade
		\item Criar reservas de valor
	\end{enumerate}}.\\
Muitas destas mudanças foram proporcionadas pelo desenvolvimento tecnológico, e se tratando de segurança,  a tecnologia de registro blockchain surgiu como base e solução como um banco de dados seguro e de registros imutáveis, tudo isso graças a internet e sua base de usuários. Nesta tecnologia já é utilizada por centenas de empresas e avanços significativos em possíveis implementações é esperada. \\
E ainda tem mais, propomos e desenvolvemos uma atividade usando os conceitos de mineração para calcular as taxas de retorno de um investimento físico para desenvolver estes recursos. 

\section{Trabalhos futuros}
Tomando então como concluídos os desafios e objetivos aqui propostos tendo em vista os resultados aqui obtidos, sugere-se que em trabalhos posteriores a este:
\begin{itemize}
	\item Sejam atualizadas todas as definições sobre a aceitação das criptomoedas no mundo, indicando o surgimento de novas melhorias no sistemas utilizados, se existerem novas criptomoedas relevantes para o cenário econômico mundial e se algumas delas caiu em desuso.
	\item Desenvolver e relatar novas propostas para o aumento da literacia em criptoativos em meio as populações mundiais, utilizando a proposta de pesquisa aqui disposta para definir novos censos neste tema, medindo e comparando os dados dispostos atuais com os dados obtidos futuramente.   
	\item Retratar sobre as consequências que a baixa  literacia em criptoativos já tem causado na população geral, sendo propensa a atuação de criminosos causando prejuízos por meio da aplicação  de golpes com esse tema, resultando em valores transferidos de forma fraudulenta para suas carteiras.
\end{itemize}   
 


