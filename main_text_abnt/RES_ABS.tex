
\bookmark[page=9,level=0]{Resumo}

\begin{KeepFromToc}
 \chapter*{RESUMO}

%Placeholder: Texto do Resumo deverá ser desenvolvido depois e aparecerá aqui
\nohyphens{
O próposito do presente instrumento é disseminar a criptoliteracia que é o conhecimento sobre criptomoedas, argumentando sobre suas funções e características com uma abordagem simplificada e compreensão desembaraçada. Assim, a história das transações humanas é resumida em um breve contexto histórico, é feita uma introdução aos principais conceitos da criptografia que é a base do sistema, e uma ideia geral sobre o que são criptomoedas e uma percepção geral da população mundial sobre o tema é exibido. Além disso, uma proposta de atividade é totalmente desenvolvida. A atividade busca mostrar de forma prática os ganhos e perdas na mineração de criptomoedas por meio do cálculo do tempo de retorno adequado do investimento em hardware on-premise, tudo como resultado de grandezas de potência computacional em conversão de gasto de energia e conceitos primários de matemática financeira e elementar. O autor espera que um dia, em um futuro próximo, os investimentos em criptoativos sejam uma opção mais acessível, simples de aplicar e compreender para quem tem vontade de investir, independente de renda, condição social, escolaridade e sexo.\\
}

\noindent
Palavras-chave: Criptomoedas; Meios de Pagamento; Criptografia; Mineração.

\vspace{2cm}

\bookmark[page=10,level=0]{Abstract}

\chapter*{ABSTRACT}
\nohyphens{
The purpose of this instrument is to disseminate cryptoliteracy, which is the knowledge about cryptocurrencies, arguing about its functions and characteristics with a simplified approach and disentangled understanding. Accordingly, the history of human transactions is summarized in a brief historical context, an introduction to the main concepts of cryptography which is the system basis is done, and a general idea about what} cryptocurrencies are and a general perception of the world population about the topic is displayed. Furthermore, an activity proposal is fully developed. The activity seeks to show in a practical way the gains and losses in cryptocurrency mining through calculating the payback time appropriate on-premise hardware investment, all as a result of computational power quantities into energy expenditure conversion and primary concepts of financial and elementary mathematics. The author hopes that one day, in the near future, investments in cryptoactives will be a more accessible option, simple to apply and comprehend for those who have the desire to invest, regardless of income, social status, educational status and sex.\\
%}

\noindent
Keywords: Cryptocurrencies; Payment methods; Cryptography; Cryptocurrencies Mining.

\end{KeepFromToc}