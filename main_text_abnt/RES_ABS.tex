
\bookmark[page=9,level=0]{Resumo}

\begin{KeepFromToc}
 \chapter*{RESUMO}

%Placeholder: Texto do Resumo deverá ser desenvolvido depois e aparecerá aqui

O próposito do presente instrumento é disseminar a criptoliteracia que é o conhecimento sobre criptomoedas argumentando sobre suas funções e características com uma abordagem simplificada e compreensão facilitada. Para isto acontecer, sintetiza-se então a história das transações humanas em um breve contexto histórico, faz-se uma introdução aos conceitos principais da criptografia, que é a base dos sistema, e uma ideia geral sobre o que são as criptomoedas e a percepção geral da população mundial sobre o tema é exibida. Mais ainda, uma proposta de atividade temática é desenvolvida por completo. A atividade procura mostrar de forma prática o custos e ganhos na mineração de criptomoedas calculando o tempo de retorno de investimento em um hardware on-premise apropriado, tudo isto por meio da conversão de grandezas de poder computacional em gasto energético e conceitos primários da matemática financeira e elementar. O autor espera que um dia, em um futuro oportuno, os investimentos em criptoativos sejam uma opção mais acessível e de fácil aplicação e compreensão para aqueles que possuem o desejo em investir, independente de renda, status social, status educacional e sexo. \\

\noindent
Palavras-chave: Criptomoedas; Meios de Pagamento; Criptografia; Mineração.

\vspace{2cm}

\bookmark[page=10,level=0]{Abstract}

\chapter*{ABSTRACT}

The purpose of this instrument is to disseminate cryptoliteracy, which is the knowledge about cryptocurrencies, arguing about its functions and characteristics with a simplified approach and facilitated understanding. For this to happen, the history of human transactions is summarized in a brief historical context, an introduction to the main concepts of cryptography is done, which is the basis of the system, and a general idea about what cryptocurrencies are and the general perception of the world population about the topic is displayed. Furthermore, a proposal for thematic activity is fully developed. The activity seeks to show in a practical way the costs and gains in cryptocurrency mining by calculating the payback time on investment in an appropriate on-premise hardware, all that through the conversion of computational power quantities into energy expenditure and primary concepts of financial and elementary mathematics. The author hopes that one day, in the future, investments in cryptoactives will be a more accessible option that is easy to apply and understand for those who have the desire to invest, regardless of income, social status, educational status and gender.\\

\noindent
Keywords: Cryptocurrencies; Payment methods; Cryptography; Cryptocurrencies Mining.

\end{KeepFromToc}