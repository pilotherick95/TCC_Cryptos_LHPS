% !TeX encoding = UTF-8
% !TeX spellcheck = pt_BR

%Intro começa aqui
\pagestyle{fancy}
\fancyhf{}
\rhead{Lucas H. P. da Silva}
\lhead{\leftmark}
\rfoot{\thepage}
\setcounter{chapter}{0}



\chapter{Introdução}


\begin{flushright}
	\begin{minipage}{8cm}
		\textit{	\noindent 
			\begin{flushright}
				``(O) Bitcoin fará aos bancos o que o e-mail fez à indústria postal'' - Rick Falkvinge (tradução livre)
		\end{flushright}}
	\vspace{1cm} 
	\end{minipage}
\footnote{Citação original do inglês: ``Bitcoin will do to banks what email did to the postal industry''}
\end{flushright}

Neste capítulo será feita uma breve introdução contextualizada, e falaremos sobre os objetivos e estruturas do documento.

\section{Contextualização}

A cada ano novas tecnologias tem tomado as prateleiras dos mercados de eletrodomésticos, eletroportáteis, a integração das tecnologias de \textit{machine learning} (ML)\footnote{Aprendizado de máquina, em tradução livre do inglês. Machine Learning trata do desenvolvimento e da utilização de sistemas computacionais que tenham a capacidade de aprender e se adaptar por conta própria, ou sejam, sem seguir instruções explícitas, por meio do treinamento de algoritmos e modelos estatísticos no intuito de obter inferências dentro de padrões em dados.} com inteligência artificial em \textit{smart devices}\footnote{Aparelhos inteligentes, em tradução livre do inglês. Smart devices são dispositivos  eletrônicos, que possuem conectividade com outros dispositivos pelos mais diferentes protocolos de conexão, como WiFi, Bluetooth, 3G, 4G, 5G, NFC, etc, que operando e possibilitando interações com os usuários e também de forma independente.}, que podemos citar como exemplo os assistentes pessoais, carros autônomos, a computação com \textit{internet of things} (IoT)\footnote{Internet das coisas, em tradução livre do inglês. São aparelhos que possuem a capacidade de transferir dados em uma rede sem a necessidade de interação homem para homem ou computador para homem. A definição de aparelhos inteligentes tem conexão direta com a internet das coisas.}. Seria evidente a criação de uma tecnologia ou moeda que reduzissem os trâmites em todas as relações financeiras que ocorrem diariamente em nossa sociedade.

O foco deste instrumento é baseado em criptoativos, uma tecnologia que tem alterado os padrões do mundo das finanças e já faz parte do mundo digital. Mas do que se tratam os criptoativos, as criptomoedas? \index{Criptomoeda}Uma criptomoeda é uma moeda digital que pode ser usada para obtenção de bens e acquisição de serviços que baseia sua segurança em  criptografia para proteger as transações online. Em sua variação de maior popularidade, o \index{Bitcoin} \textit{Bitcoin (BTC)}, devido a sua especulação, tem enfrentado uma consequente volatilidade. Por exemplo, segundo dados do \cite{COINBASE}, em 11 de dezembro de 2020, 1 BTC valia R\$ 91.415,48, três meses depois, R\$ 321.078,18 e mais três depois voltou ao patamar de R\$ 191.042,90. 

%Muito do interesse nessas moedas não regulamentadas é o comércio com fins lucrativos, com especuladores às vezes elevando os preços para o alto 

\section{Temática}
Caro leitor, tome o seguinte momento para questionar sobre o progresso tecnológico que ocorreu nos últimos anos. Pondere sobre como a tecnologia tem influenciado tua vida e daquele que lhe cercam e verás então como será importante o conhecimento sobre criptoativos nos dias futuros. Chris Garrod, em uma reportagem do site Conyers cita o seguinte:  
 

\begin{citacao}
	Os mercados de criptomoedas estão em todos os lugares. $(\cdots)$ Não são ouro e nem dinheiro fiduciário \footnote{Veja mais sobre dinheiro fiduciário em \ref{fiat}}. Esta é uma tecnologia totalmente nova que já ilustrou sua capacidade de perturbar fundamentalmente o sistema financeiro global. \cite{GARROD}
	\end{citacao}

De fato, manter-se atualizado sobre o momento atual da tecnologia é fundamental. Considere a seguinte analogia:  A moeda na era digital, se comparados aos meios de comunicação, é como ao e-mail versus o serviço físico postal. Antes da internet, as pessoas dependiam dos serviços postais para enviar uma mensagem a quem estivesse em outro lugar, era preciso um intermediário para entregá-la fisicamente, inimaginável para os possuintes do precursor da comunicação eletrônica e ainda mais inimaginável aos possuintes de seus sucessores, os serviços de mensageria criptografada. Isto é o que a criptomoeda se comparará ao dinheiro físico, perturbando fundamentalmente o sistema financeiro global, se observarmos a citação de \cite{GARROD}. 

\section{Objetivos} \label{objetivos}
\subsection{Objetivo Geral}
Sintetizar a história das transações humanas em um breve contexto histórico e disseminar criptoliteracia que é o conhecimento sobre criptomoedas argumentando sobre suas funções e características na esperança de em algum momento no futuro estes sejam acessíveis e de fácil compreensão, independentemente da renda, status educacional ou sexo.  \index{Criptoativo}

\subsection{Objetivos Específicos}

\begin{itemize}
	\item Desenvolver um breve histórico da evolução das transações humanas 
	\item Introduzir sobre a história da criptografia e sobre como ela é a base das moedas virtuais.
	\item Trazer o conceito de criptoativo tomando nota de sua importância no cenário econômico atual.  
	\item Propor uma atividade temática que mostre aos alunos os custos e lucros de se minerar criptoativos. 
\end{itemize}

\section{Estruturação do trabalho}
O presente documento foi desenvolvido em quatro partes, permitindo assim que o prezado leitor tenha uma melhor separação de informações aqui prestadas e com isso uma melhor compreensão dos 
objetivos definidos no setor acima: 
\begin{itemize}
	\item No prólogo é apresentado uma introdução geral do tema, por meio de uma
	contextualização, juntamente com as definições tomadas. 
	\item No segundo capítulo, é apresentado um resumo histórico das evoluções de transações e das moedas, como a utilização dos metais, evolução do papel moeda, cartões de débito e crédito até o dinheiro digital. Esta seção prepara o leitor para os criptoativos, tema do próximo capítulo.
	\item No terceiro capítulo, apresentaremos o futuro das moedas. Será feita uma introdução rápida sobre criptografia, uma ideia geral sobre o que são as criptomoedas e a percepção geral da população mundial sobre o tema. Esta seção prepara o leitor para a atividade do próximo capítulo.
	\item No quarto capítulo é desenvolvida uma atividade que mostre aos alunos o custos e ganhos na mineração de criptomoedas, além do cálculo do tempo de retorno do investimento de um hardware apropriado.
	\end{itemize}

\section{Relevância do estudo}

A pesquisa pelo tema o que se apresenta este presente documento se deu inicialmente na boa relação com o Prof Dr. Mohammad Fanaee. Em várias conversas durante os intervalos de suas aulas fui convidado a participar dos estudos orientados que ele também ministrava e palestras que ele participava. Uma destas palestras, ministrada pelo grupo \index{Evolucionários} Evolucionários da Universidade Federal Fluminense, cujo tema foi "Investimento de renda fixa e criptoativos", foi o marco inicial do interesse pelo tema. Nesse mesmo período, final de 2018, as moedas digitais já tinham um grande crescimento e conhecimento da mídia, principalmente o Bitcoin,\index{Bitcoin} que trouxe ao grupo de conversas diversos questionamentos sobre como estes aparatos tecnológicos poderiam impactar as economias tanto locais quanto internacionais. \\
Apesar de sua alta volatilidade e serem definitivamente recentes, as criptomoedas já estão cada vez mais aparentes nos veículos de comunicação, com uma base de usuários crescente, com mais transações sendo realizadas diariamente, sem contar a vantagem de ser globalizada e ter como característica a descentralização sem fronteiras políticas tendo ativo desenvolvimento. 

 