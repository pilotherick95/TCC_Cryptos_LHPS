\chapter*{Glossário}
\addcontentsline{toc}{chapter}{Glossário}
\chaptermark{Glossário}
\pagestyle{empty}

\begin{center}
	%\begin{fakeminipage}
	\begin{itemize}
		\item[Asset] Assets são conjuntos de objetos disponíveis para desenvolvimento de exemplos de aplicações, imagens, vídeos em diversas plataformas. Especificamente falando do \cite{WHIM}, os assets são conjuntos de ícones e setas para o desenho de infográficos, mapas mentais e post-its.
		 
		\item[AMD] Fabricante de GPUs AMD, cujo nome vem do inglês e significa \textit{Advanced Micro Devices} ou "Micro aparelhos avançados". É uma empresa multinacional americana de semicondutores com sede em Santa Clara, Califórnia, que desenvolve processadores de computador e outros tipos de hardware. 
		
		\item[Backup] Um backup de um certo dado é uma cópia de segurança deste dado, porém obtido e armazenado em outra localização para frutos de restauração em caso de perda do dado original ou  acidente 
		
		\item[Ethash] Ethash é o algorítmo de mineração oficial para ETH. Trata-se de um algoritmo de alta qualidade que usa elaboradas técnicas computacionais para garantir a maior segurança possível na mineração.
		
		\item[Ethash4G] Ethash4G é uma derivação do ethash e permite que GPUs com 4GB de VRAM consigam minerar.
		
		\item[Fintech] Junção das palavras financial e technology (financeiro e tecnologia) do inglês. São startups nos quais o uso da tecnologia é o principal diferencial, se comparados com outras empresas do mesmo ramo. Desenvolvem produtos financeiros totalmente digitais. 
		
		\item[Frame] Os Frames são gabinetes especializados para mineração e são compostos por estrutura metálicas dando suporte físico para todo hardware de mineração.
		
		\item[Hands on] Trata-se de uma expressão muito utilizada no mundo corporativo que refere-se equivalentemente a expressões do tipo "mão na massa" ou "aprender enquanto pratica" quando uma demanda ou atividade é aplicada. Também pode significar pró-atividade por parte de um funcionário ou aluno.
		
		\item[Hardware] Conjunto de equipamento físico de qualquer sistema gerenciado computacionalmente. Constituem-se de peças e equipamentos que fazem influenciam na velocidade de processamento e capacidade de armazenamento de um sistema, além de outras funções
		
		\item[Kawpow] Kawpow é o algorítmo de mineração oficial para RVN, que foi implementado em 6 de maio de 2020.
		
		\item[Marketplace] No mundo físico trata-se de um espaço aberto onde um mercado ou feira é ou foi realizado anteriormente em uma cidade. No mundo digital, são páginas programadas e especializadas na compra,venda e troca de objetos.
		
		\item[MTP] MTP (Merkle Tree Proof ou Provas para árvores de Merkle em português) é um algoritmo de mineração PoW usado por Zcoin que mudou para MTP em 10 de dezembro de 2018.
		
		\item[Nvidia] Fabricante de GPUs Nvidia, cujo nome vem da palavra \textit{Invidia}, do latim, que significa inveja. É uma empresa multinacional americana de tecnologia sediada em Santa Clara, Califórnia que desenvolve GPUs para mercados de jogos e profissionais.
		
		\item[Open-source] Originado na computação, 
		O termo open-source se refere a algo que as pessoas podem modificar e compartilhar pois seu design é acessível ao público.
				
		\item[PySpark]  O nome Pyspark se dá pela junção de Python com Apache Spark. Este é considerado uma interface para Apache Spark em Python. Com ele, é possível escrever aplicativos usando APIs Python e é amplamente difundido em grandes empresas por ser open-source.
		
		\item[Rig] Rigs de mineração são o conjunto de sistemas computacionais estruturados para mineração. Este inclui placa-mãe, fonte, memória RAMs, FRAME e RISERs. Sua função é controlar e realizar suporte sobre o processamento realizado por GPUs de alto desempenho.
		
		\item[Riser] Risers são conjunto de cabeamentos elétricos e adaptadores que tem o objetivo de conectar GPUs a outros componentes de um RIG. Tais elementos utilizam um frame para permanecerem em posição.  
		
		\item[Software] Conjunto de partes lógicas de qualquer sistema gerenciado computacionalmente.	Sua função é executar, realizar instruções e manipular as atividades lógicas por meio de um hardware. Divide-se em softwares de sistemas que permite interações dos usuários e o hardware e software de aplicação, que permitem que o usuário realize atividades específicas como criação de gráficos, escrita de textos e manipulação de planilhas.
		
		\item[Startup] Termo inglês que se traduz como o início de algo novo. São empresas que estão em fase inicial e possuem novos conceitos para mostrar ao mercado.
		
	\end{itemize}
	
%\end{fakeminipage}
\end{center}





