% !TeX encoding = UTF-8
% !TeX spellcheck = pt_BR

% ---
% Agradecimentos
% ---
\begin{agradecimentos}
	
\noindent
Ao único e poderoso Deus, que me deu a oportunidade de chegar até aqui, dando-me propósito, perdoando minhas falhas, guiando-me, preservando minha fé, que deu-me uma formação espiritual suficientemente forjada para perseverar e prevalecer contra os pecados e tentações tão amplamente disponíveis no ambiente universitário. Deus não está morto, certamente vivo está!\\

\noindent
Ao meu núcleo familiar, meus amados pais, digníssimos Sra. Francisca e Sr. Jozivan Pereira. Estes foram responsáveis por serem exemplos de rigor, sabedoria, integridade e carinho para minha formação, governando-me no caminho que devo andar, em disciplina e admoestação, provendo sabedoria nos momentos de adversidade, ensinando-me sobre as santas escrituras e disciplinando quando necessário. Esta conquista é vossa e também todas as outras que ainda hão de surgir \heart \\

\noindent
A minha amada irmã Sra. Francine Oliveira, que me fez relembrar que a matemática não está limitada somente ao escopo da docência, mas sendo também a base do mundo corporativo. Agradeço por seus ensinamentos dentro da Engenharia e Análise de Dados, \textit{Business Intelligence} e sistemas Linux. Estes me fizeram desenvolver interesse nas mais diversas \textit{libraries}(\textbf{\textit{Libs}}), e estruturas de programação como \textit{Structured Query Language} (\textbf{SQL}) e \textit{Python}/\textit{Python Spark} (\textbf{PySpark}).  \\

\noindent
Aos meus queridos tios, Sra. Marcia e Sr. Jary Amorim, que por sua intervenção me permitiram prosseguir com os estudos. \\

\noindent
Aos professores, em especial aqueles que possibilitaram escrever parte da minha história dentro do instituto de matemática e estatística da universidade federal fluminense. São eles: Fábio Henrique e Roberto Geraldo do departamento de geometria, Ricardo Apolaya, Renata de Freitas, Miriam Abdon, Anne Michelle e Paulo Trales do departamento de análise, Carlos Mathias, Humberto Bortolossi e Sebastião Firmo (in memoriam) do departamento de matemática aplicada. Mestres, dou-lhes minha admiração e respeito, saúdo-vos e agradeço-vos pela maestria em suas atuações. \\

\noindent
Ao professores Fabiano Souza e Sandra William, que foram meu mentores na primeira seleção do Programa Institucional de Residência Pedagógica. Com vossas permissões consegui desenvolver e aplicar diversos projetos tecnológicos para muitos alunos das mais diversas turmas de matemática, projetos que variam de aplicações em simuladores de voo até protótipos de aplicativos android compilados em C\# e Unity.\\  

\noindent
Ao meu orientador Prof Dr. Mohammad Fanaee por me receber como orientado em seu gabinete. Agradeço imensamente pela confiança no meu trabalho, pela compreensão,pelo respeito, por seus excelentes ensinamentos não só de matemática mas de vivência por meio de sábios conselhos em vários momentos da graduação.\\

\noindent
A esta pequena lista dos mais memoráveis camaradas da jornada universitária:  Alexandre Lopes, Everaldo Costa, Lair Júnior, Leandro Goiano, Janaína Citeli, Jéssica Cecília da química, Matheus Gama, Nayara Lima, Renan Santo, Raphael Odalvo, Victor Chirity e Victor Sarmet. Agradeço pelas tardes e noites viradas de estudo, pelos almoços e jantares no restaurante universitário, idas e vindas nas bibliotecas da universidade e pelas incontáveis vezes no qual a vida se tornou mais significativa com vós. Camaradas, permanecei firmes, pois as provações são grandes, mas a vitória (a formatura!) é iminente. \\

\noindent
Finalmente, meu agradecimento a todos aqueles que não tiveram menção direta mas tive o prazer de conhecer, que pude ajudar, que recebi alguma ajuda e acompanhei direta ou indiretamente, expandindo meu networking acadêmico e profissional.\\ 
 
\noindent
Obrigado. Thank you. Merci. Gracias. 감사합니다 {\tiny(gamsahamnida)}.



	
\end{agradecimentos}
% ---