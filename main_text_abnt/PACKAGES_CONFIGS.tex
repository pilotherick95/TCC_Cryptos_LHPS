
%Texto de TCC2 de Lucas Herick Silva
%https://www.linkedin.com/in/lucasherick/

\usepackage[utf8]{inputenc}
\usepackage[T1]{fontenc}
\usepackage{lmodern} % fonte latin modern
\usepackage{imakeidx} \makeindex[columns=2] %
\usepackage{titlesec} %Fontes - regulamentação=
\usepackage{amsmath,amssymb,amsthm} %amsfonts?
\usepackage{tabularx}	%Entrada em Tabelas
\usepackage{pdfpages}
\usepackage{float}
\usepackage{url} %links
\PassOptionsToPackage{hyphens}{url}
\usepackage{wrapfig}
\usepackage{multicol}
\usepackage{multirow}
\usepackage{array}
\usepackage{longtable}  %tabelas em duas páginas
\usepackage{booktabs}
\usepackage{indentfirst}% Indenta 1o parágrafo da seção.
\usepackage{color}		% Controle das cores
\usepackage{graphicx}	% Inclusão de gráficos
%\usepackage{subcaption}% Inclusão de Sub-legendas
\usepackage{subfig}		% Imagens lado a lado	
\usepackage{microtype} 	% para melhorias de justificação
\usepackage{hyphenat}
\usepackage{multirow}
%\usepackage{setspace}
\usepackage{xcolor}		%Allows colored text 
\usepackage[most]{tcolorbox} %Caixa preta ou colorida 
\usepackage[left=3cm, right=2cm, top=3cm, bottom=2cm]{geometry}

\usepackage{mathtools}
\newcommand\underrel[3][]{\mathrel{\mathop{#3}\limits_{%
			\ifx c#1\relax\mathclap{#2}\else#2\fi}}}

\renewcommand{\baselinestretch}{1.3}

% informações do PDF
\hypersetup{
	%pagebackref=true,
	pdftitle={\@title}, 
	pdfauthor={\@author}, 
	colorlinks=false,       	% false: boxed links; true: colored links
	linkcolor=blue,          	% color of internal links
	citecolor=blue,        		% color of links to bibliography
	filecolor=magenta,      		% color of file links
	urlcolor=blue,
	bookmarksdepth=4}


%-------------------------------------------------------------

%ABNT SETUP
% Pacotes de citações
% ---
\usepackage[brazilian,hyperpageref]{backref}	 % Paginas com as citações na bibl
\usepackage[alf]{abntex2cite}	% Citações padrão ABNT
% ---

% ---
% Configurações do pacote backref
% Usado sem a opção hyperpageref de backref
\renewcommand{\backrefpagesname}{Citado na(s) página(s):~}
% Texto padrão antes do número das páginas
\renewcommand{\backref}{}
% Define os textos da citação
\renewcommand*{\backrefalt}[4]{
	\ifcase #1 %
	Nenhuma citação no texto.%
	\or
	Citado na página #2.%
	\else
	Citado #1 vezes nas páginas #2.%
	\fi}%

\chapterstyle{default} %default %ell %southall
\renewcommand{\ABNTEXpartfont}{\normalfont}
\renewcommand{\ABNTEXsectionfont}{\normalfont}
\renewcommand{\ABNTEXsubsectionfont}{\normalfont}
\renewcommand{\ABNTEXsubsubsectionfont}{\normalfont} 

\usepackage{fancyhdr}

%---------CAPA PACKAGES-------
\usepackage{eso-pic}

\usepackage[overlay,absolute]{textpos}
\newcommand\PlaceText[3]{%
	\begin{textblock*}{10in}(#1,#2)  %% change width of box from 10in as you wish
		#3
	\end{textblock*}
}%
\textblockorigin{-5mm}{0mm}% Default origin top left  
%---------------------------------------

%---------INTRO PACKAGES-------
%\usepackage{tocloft}
\usepackage{etoc}
\etocsettocstyle{\chapter*{\normalfont\bfseries SUMÁRIO}}{}

%-------AGRAD PACAKGES------------------
\usepackage{wasysym}
\usepackage{harmony}
\newcommand{\heart}{\ensuremath\heartsuit}
\newcommand{\butt}{\rotatebox[origin=c]{180}{\heart}}
\usepackage{kotex} 

%------CHAPTERS 1-10 PACKAGES--------
%\usepackage{coloremoji}
\renewcommand{\chaptername}{Capítulo}

%-----GLOSSARIO-----
\usepackage{hyperref}
\usepackage{marginnote}
%\newtcolorbox{fakeminipage}[1][]{enhanced jigsaw,breakable,boxsep=0pt,boxrule=0pt,width=\textwidth,colback=white,#1}

%------REF BIBLIOGRAFS PACKAGES-------
%\usepackage{bibspacing}
%\setlength{\bibitemsep}{2\baselineskip}

\renewcommand{\bibsection}{\chapter*{Referências Bibliográficas}}

\usepackage{lastpage}
%Texto de TCC2 de Lucas Herick Silva
%https://www.linkedin.com/in/lucasherick/

%%%%%%%%%%%%%%%%%%%%%%%%%%%%%%%%%%%%%%%%%%%%%%%%%%%%%%